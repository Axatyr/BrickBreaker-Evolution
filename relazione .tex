\documentclass[a4paper,12pt]{report}

\usepackage{graphicx}
\usepackage[utf8]{inputenc}
\usepackage{hyperref}

\title{BrickBreaker Evolution}

\author{Alessandro Palladino, Alex Testa, Francesco Raso, Giacomo Totaro}
\date{A.A 2020-2021}

\begin{document}

\maketitle

\tableofcontents %inserisce indice

\chapter{Analisi}
\section{Requisiti }
    BrickBreaker Evolution è un gioco ispirato al celebre arcade "Brick Breaker", ma in una veste innovativa.
    \\
    L'utente avrà a disposizione un numero limitato di vite, una barra e una pallina che dovrà far rimbalzare per distruggere i mattoncini. Lo scopo del gioco sarà riuscire a distruggere tutti i blocchi per poter avanzare a livelli sempre più impegnativi, bisognerà però stare attenti alle insidie che si celano all'interno dei mattoncini che potranno renderti il gioco più facile o più difficile a seconda della fortuna(powerup). Il giocatore perderà una vita quando la pallina colliderà con il bordo inferiore dell'area di gioco. Al termine delle vite il punteggio finale verrà registrato in classifica.
    \\
    L'utente inoltre potrà selezionare varie modalità di gioco che avranno diversi temi ispirati a giochi retrò e funzionalità per rendere il gioco ancora più impegnativo.
    
    \subsection*{Requisiti funzionali}
        \begin{itemize}
            \item All'inizio della partita verrà generata una barra, una pallina e un numero limitato di vite. 
            \item La barra può essere spostata solo orizzontalmente. 
            \item La pallina potrà rimbalzare sulle pareti dell'area di gioco (esclusa quella inferiore), sui mattoncini e sulla barra; a seconda dell'angolazione la pallina assumerà traiettorie diverse.
            \item Le vite potranno essere selezionate (in base alla difficoltà scelta) nel menù di gioco e diminuiranno ogni volta che la pallina collide con il bordo inferiore. Al termine delle vite la partita terminerà.
            \item Al momento della distruzione di alcuni mattoncini verranno generati casualmente dei potenziamenti che potranno favorire od ostacolare la partita.
            \item All'interno del menù sarà possibile scegliere con quale tema giocare e di conseguenza cambierà l'area di gioco.
            \item A fine partita l'utente potrà inserire il proprio nome e visualizzare
            il relativo punteggio conseguito all'interno di una classifica generale, sarà inoltre possibile consultare la classifica all'interno del menù.
        \end{itemize}
    
    \subsection*{Requisiti non funzionali}
        \begin{itemize}
            \item Gestire i dati persistenti tramite salvataggio su file.
            \item Implementazione della fonica nel menu e durante la partita.
            \item Creazione automatizzata dei livelli generata in modo randomico.
        \end{itemize}
    
\section{Analisi e modello del dominio}

L'entità principale del modello applicativo è la Board, che rappresenta il quadrante di gioco in cui vi sono tutti i componenti rappresentati dai GameObject. Quest'ultimi rappresentano le parti fondamentali che compongono la mappa di gioco e sono: i mattoncini(Brick), una pallina(Ball), i powerup(PowerUp) e la barra (Paddle), impersonata dal giocatore(Player).
\\
Gli elementi costitutivi il problema sono sintetizzati in %\Cref{SchemaUML}.
\\
Le difficoltà principali saranno gestire le collisioni tra le entità e generare randomicamente powerup e livelli.
\\
Data la complessità del progetto, in questa prima versione non verrà implementata la feature relativa agli easter egg.


\begin{figure}[h]
\centering{}
 \includegraphics[scale=0.70]{SchemaUML.png}
\caption{Schema UML dell'analisi del problema, con rappresentate le entità principali ed i rapporti fra loro}
\label{SchemaUML}

\includegraphics[scale=0.70]{Diagramma.png}
\caption{Diagramma a stati del videogioco}
\label{Diagramma}

\end{figure}

\chapter{Design}
\section{Architettura}
\section{Design dettagliato}
    \subsection*{Alessandro Palladino}
    
    
    \subsection*{Alex Testa}
    
    
    \subsection*{Francesco Raso}
    
    
    \subsection*{Giacomo Totaro}
    -realizzazione e gestione della pallina e paddle, usando pattern builder e costruendo i relativi componenti input, fisici e grafici.
    -gestione delle collisioni in accordo con Francesco, Le collisioni vengono rilevate dal collision manager, che calcola tutte le varie collisioni che possono presentarsi nel gioco. La board comunica la collisione ai GameObjcet che, se collidono, notificano l'evento alla Board. Quest'ultimo verrà aggiunto in una cosa e sarà un istanza dell'interfaccia Event. Questa coda di eventi verrà risolta dall'eventManage() ad ogni update della Board
    

\chapter{Sviluppo}
\section{Testing automatizzato}
    \subsection*{Alessandro Palladino}
    
    
    \subsection*{Alex Testa}
    
    
    \subsection*{Francesco Raso}
    
    
    \subsection*{Giacomo Totaro}
    -Ho testato il paddle, quindi creazione con il relativo builder, verificato se prendesse corretamente gli input da tastiera e che si muovessere di una quantità dipendente dalla velocità dello stessi e dalla frequenza di aggiornamento importata dal gameLoop controller.
    -testato il corretto funzionamento delle collisioni, insieme al mio collega Francesco, e della comunicazione di queste alle entità in causa. maggiore attenzione al controllo delle collisioni con il paddle per verificare il calcolo della direzione della pallina.
    -testato la pallina, verificando la creazione attraverso il builder, e verificando il corretto aggionramento del componente fisico aggiornandolo in modo da controllare che la posizione risultante dopo l'aggiornamento fosse quella giusta. Il controllo sulla velocità è stato eseguito nello stesso modo.
    
\section{Metodologia di lavoro}
    \subsection*{Alessandro Palladino}
    
    
    \subsection*{Alex Testa}
    
    
    \subsection*{Francesco Raso}
    
    
    \subsection*{Giacomo Totaro}
    
    
\section{Note di sviluppo}
    \subsection*{Alessandro Palladino}
    
    
    \subsection*{Alex Testa}
    
    
    \subsection*{Francesco Raso}
    
    
    \subsection*{Giacomo Totaro}

\chapter{Commenti finali}
    \subsection*{Alessandro Palladino}
    
    
    \subsection*{Alex Testa}
    
    
    \subsection*{Francesco Raso}
    
    
    \subsection*{Giacomo Totaro}
    
    
\section{Autovalutazione e lavori futuri}
    \subsection*{Alessandro Palladino}
    
    
    \subsection*{Alex Testa}
    
    
    \subsection*{Francesco Raso}
    
    
    \subsection*{Giacomo Totaro}
    
    
\section{Difficoltà incontrate e commenti per i docenti}
    \subsection*{Alessandro Palladino}
    
    
    \subsection*{Alex Testa}
    
    
    \subsection*{Francesco Raso}
    
    
    \subsection*{Giacomo Totaro}
    
    

\appendix
\chapter{Guida utente} 
%vedremo se implementare un mini tutorial oppure no


\end{document}
